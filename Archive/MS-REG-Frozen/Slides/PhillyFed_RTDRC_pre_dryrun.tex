
% \documentclass[notes=show]{beamer}
\documentclass[xcolor=dvipsnames, xcolor=table, 10pt]{beamer}
\setbeamercovered{dynamic}
%%%%%%%%%%%%%%%%%%%%%%%%%%%%%%%%%%%%%%%%%%%%%%%%%%%%%%%%%%%%%%%%%%%%%%%%%%%%%%%%%%%%%%%%%%%%%%%%%%%%%%%%%%%%%%%%%%%%%%%%%%%%%%%%%%%%%%%%%%%%%%%%%%%%%%%%%%%%%%%%%%%%%%%%%%%%%%%%%%%%%%%%%%%%%%%%%%%%%%%%%%%%%%%%%%%%%%%%%%%%%%%%%%%%%%%%%%%%%%%%%%%%%%%%%%%%

\usepackage{amsmath}
%\usepackage{mathpazo}
\usepackage{natbib}             % Option: Use NatBib bibliography styles
\usepackage{hyperref}
\usepackage{multimedia}
\usepackage{graphicx}
\usepackage{helvet}
\usepackage[english]{babel}
\usepackage[latin1]{inputenc}
\usepackage{comment}
\usepackage{color}
\usepackage{epstopdf}
\usepackage{appendixnumberbeamer}
\usepackage{transparent}
\usepackage{xcolor}
\usepackage{relsize}
\usepackage{booktabs}
\usepackage{amsthm}
\usepackage{threeparttable, tablefootnote} % table footnote
%\usepackage{PlayfairDisplay}
\usepackage{caption}
\DeclareMathOperator*{\argmin}{argmin} % thin space, limits underneath in displays
%\usepackage{bbm}

\setcounter{MaxMatrixCols}{10}

% \USEINNERTHEME{CIRCLESu}
%\usefonttheme{serif}
% \usecolortheme{orchid}

\usecolortheme[RGB={26,58,95}]{structure}
\beamertemplatenavigationsymbolsempty

\definecolor{orange}{RGB}{255,127,0}
\definecolor{upf}{RGB}{192,0,37}
\definecolor{euiblue}{RGB}{56,136,199}
\definecolor{fedblue}{RGB}{26,58,95}
\definecolor{darkred}{rgb}{0.55, 0.0, 0.0}
\definecolor{applegreen}{rgb}{0.55, 0.71, 0.0}
\definecolor{orange}{rgb}{0.93, 0.53, 0.18}
\definecolor{pennblue}{RGB}{26,58,95}
\definecolor{darkgreen}{rgb}{0.0039    0.1211    0.3555}
\newcommand{\bb}[1]{{\color{euiblue}#1}}
\newcommand{\fb}[1]{{\color{fedblue}#1}}
\newcommand{\pb}[1]{{\color{fedblue}#1}}
\newcommand{\gre}[1]{{\color{applegreen}#1}}
\newcommand{\rr}[1]{{\color{darkred}#1}}
\newcommand{\orr}[1]{{\color{orange}#1}}
\newcommand{\gr}[1]{{\color{darkgreen}#1}}

\usepackage{mathtools}

 \newcommand{\lefta}{ \left[\begin{array} }
\newcommand{\righta}{ \end{array} \right]}

\newcommand*{\vcenterimage}[1]{\vcenter{\hbox{\includegraphics[width=0.35\linewidth]{#1}}}}
\newcommand*{\vcenterarrow}{\vcenter{\hbox{$\Longrightarrow$}}}
\newcommand*{\h}{\hspace{0.15cm}}

\makeatletter
\def\blfootnote{\gdef\@thefnmark{}\@footnotetext}
\makeatother


\useitemizeitemtemplate{%
    \raise1.5pt\hbox{\color{beamerstructure}$\bullet$}%
}
\usesubitemizeitemtemplate{%
    \small\raise1.5pt\hbox{\color{beamerstructure}$\bullet$}%
}
\usesubsubitemizeitemtemplate{%
    \raise1.5pt\hbox{\color{beamerstructure}$\circ$}%
}


\setbeamersize{text margin left=1.5em,text margin right=1.5em}
\setbeamertemplate{footline}[frame number]

\usetheme{boadilla}


\begin{document}

%\playfair

\title[Growth-At-Risk Models]{\textbf{Growth-At-Risk Models\\ for the U.S. and the Foreign Economy Outlook}}
\thispagestyle{empty}
\author[Caldara, Cascaldi-Garcia, Cuba-Borda, Loria]{\textbf{Dario Caldara \\ Danilo Cascaldi-Garcia \\ Pablo Cuba-Borda \\ Francesca Loria }\\ \medskip \emph{Federal Reserve Board} \\ \bigskip \bigskip
\emph{2020 Conference on Real-Time Data Analysis, Methods and Applications}}

%\date[Class II - FOMC]{\emph{May 12, 2020}}
%\date[May 12, 2020]{\emph{May 12, 2020}}
\date{\today}
\maketitle

% \begin{frame}{Overview}
%  \tableofcontents
%  \end{frame}

%\setcounter{framenumber}{0}


%-----------------------------------------------------------------
\begin{frame}{Motivation}
\begin{itemize}
\item Risk management is an (increasingly) important consideration for policy decisions
\medskip
\item Quantifying risks to the economic outlook and understanding sources of risk
\medskip
\item Lively debate about measurement and sources of risks:
\medskip
  \begin{itemize}
  \item Can we reliably detect time-variation in downside risk?
  \medskip
  \item What are the drivers of downside risk?
  \end{itemize}
\end{itemize}

\end{frame}


%-----------------------------------------------------------------

\begin{frame}{Our paper}
\vspace*{0.12in}
\begin{itemize}
\item Markov-switching structure to model the \rr{entire} distribution of future real GDP growth \rr{conditional on economic activity and financial conditions}.
\medskip
     \begin{itemize}
     \item Endogenous transition probabilities depend on macroeconomic and financial conditions
\medskip
     \item Parsimonious model to capture features of growth-at-risk
\end{itemize}
\bigskip
\item MS model has a semi-structural interpretation that helps understand GAR dynamics
       \medskip
       \begin{itemize}
         \item Connects GAR literature with structural macro models
           \medskip
         \item Facilitates interpretation of Quantile Regression models
           \medskip
         \item Similar results and forecasting performance to QR models
       \end{itemize}

\end{itemize}
\end{frame}

%-----------------------------------------------------------------

\begin{frame}{The paper in one figure...}
\begin{quote}
\scriptsize{
  Financial market conditions have deteriorated, and tighter credit conditions and increased uncertainty have the \rr{potential to restrain economic growth going forward}. In these circumstances, although recent data suggest that the \gr{economy has continued to expand at a moderate pace}, the Federal Open Market Committee judges that the \bf{downside risks to growth have increased appreciably}}
\end{quote}
\vspace*{-0.5cm}
\begin{figure}
\only<1>{\includegraphics[width=0.7\linewidth]{FiguresPhillyFed/SVAR_StateDependent_2Regimes_Density_Aug2007_1.pdf}}
\only<2>{\includegraphics[width=0.7\linewidth]{FiguresPhillyFed/SVAR_StateDependent_2Regimes_Density_Aug2007_2.pdf}}\par
\only<3>{\includegraphics[width=0.7\linewidth]{FiguresPhillyFed/SVAR_StateDependent_2Regimes_Density_Aug2007_3.pdf}}\par
\only<4>{\includegraphics[width=0.7\linewidth]{FiguresPhillyFed/SVAR_StateDependent_2Regimes_Density_Aug2007_4.pdf}}\par
\end{figure}
\only<1->{  \begin{itemize}
    \only<1> { \item Macroeconomic and financial conditions influence likelihood of \rr{bad regime}}
    \only<2> { \item Intuition: predictive distribution as weighted average of \gr{normal} and \rr{bad} regimes}
    \only<4> {\item MS model consistent with staff's view and FOMC statement of \gr{benign modal outcome} but \bf{with significant downside risk}}
  \end{itemize}
}
\end{frame}



%-----------------------------------------------------------------

\begin{frame}{The paper in one figure \rr{(almost)}...}
 \begin{figure}
\includegraphics[width=0.7\linewidth]{FiguresPhillyFed/SVAR_StateDependent_2Regimes_Density_Aug2007_5.pdf}
\end{figure}
\begin{itemize}
    \item MS and QR models capture downside risk
    \item MS complementary tool for risk assessment
    \item Advantage: explicit mechanism of growth-at-risk
  \end{itemize}
\end{frame}

%-----------------------------------------------------------------

\begin{frame}{Markov switching model of GAR}
\vspace*{-0.2in}
\begin{eqnarray*}
\bar{\Delta} {y}_{t+1,t+12} &=& \alpha_y (s_t) + \beta_y (s_t) f_t + \gamma_y (s_t) m_t + \sigma_y (s_t) \varepsilon^y_t, \label{eq:MS_growth}%\\
	%f_t &=& \alpha_f + \beta_f f_{t-1} + \gamma_f m_t + \eta_{f} %m_{t-1} + \sigma_f \varepsilon^f_t, \label{eq:MS_f} \\
	%m_t &=&  \alpha_m  +\beta_{m} f_{t-1} + \eta_m m_{t-1} + \sigma_m \varepsilon^m_t, \label{eq:MS_m}
\end{eqnarray*}

\begin{itemize}
    \item $s_t=\left\{1,2\right\} \rightarrow \left\{normal, bad\right\}$: captures the unobserved economic state
\bigskip
\item $s_t$ follows Markov process with \textbf{endogenous transition probabilites}:

\begin{eqnarray*}
\mathbb{P}\left(s_{t+1} = 2 | s_t = 1\right) &=& \frac{1}{1+exp(a_{12} - b_{12} f_t - c_{12} m_t)},\\
\mathbb{P}\left(s_{t+1} = 1 | s_t = 2\right) &=& \frac{1}{1+exp(a_{21} - b_{21} f_t - c_{21} m_t)}.
\end{eqnarray*}

\end{itemize}

%\textbf{Endogenous Transitions}
\begin{itemize}
\item Financial factor $f_t$ and macroeconomic factor $m_t$ follow a VAR(1) structure
%\item Regime 1 \textit{(normal)}: High average growth, low volatility
%\item Regime 2 \textit{(bad)}: Low average growth, high volatility
%\item In the paper:
%\begin{itemize}
%\item 3-regime and additional Markov-chain for $\sigma_y$
%\item Compare to fixed transition MS model
%\end{itemize}
\end{itemize}
\end{frame}


%-----------------------------------------------------------------
\begin{frame}{The channels of GAR}
\vspace*{-0.2in}
\begin{eqnarray*}
\bar{\Delta} {y}_{t+1,t+12} &=& \rr{\alpha_y} \orr{(s_t)} + \pb{\beta_y} \orr{(s_t)} f_t + \pb{\gamma_y} \orr{(s_t)} m_t + \rr{\sigma_y} \orr{(s_t)} \varepsilon^y_t, \label{eq:MS_growth}%\\
\end{eqnarray*}

\begin{itemize}
\medskip
\item \rr{Negative correlation between mean and volatility}
\begin{itemize}
  \medskip
  \item $s_t=1$: average growth, low volatility
    \medskip
  \item $s_t=2$: low growth, high volatility
\end{itemize}
\bigskip
\item \gr{Time-variation in sensitivity of GDP growth to fundamentals}
\begin{itemize}
   \medskip
   \item Growth more sensitive to fundamentals in \textit{bad regime}
     \medskip
   \item Akin to non-linear dynamics of DSGE models $\citep{GKP2019,FernandezVillaverdeHurtadoGalo2019,ACHSV2020}$
\end{itemize}
\bigskip
\item \orr{Financial and macroeconomic conditions influence regime transitions}
\end{itemize}

\end{frame}

%-----------------------------------------------------------------

\begin{frame}{Rest of the talk}

\begin{enumerate}
    \item Real-time measurement of financial and macroeconomic conditions
    \bigskip
    \item Results Markov-switching Model
    \bigskip
    \item Comparison with Quantile Regressions
    \bigskip
    \item Risk Assessment
      \begin{itemize}
        \item GFC
         \item COVID-19
      \end{itemize}
\end{enumerate}

\end{frame}


%-----------------------------------------------------------------

\begin{frame}{Macro and financial conditions}
\begin{itemize}
    \item Summarize financial and macroeconomic conditions with a mixed-frequency Dynamic Factor Model \cite{AruobaDieboldScotti2009}
    \medskip
    \begin{enumerate}
    	\item $f_t$: VXO; Excess Bond Premium; TED spread; CBill Spread.\medskip
    	\item $m_t$: IP, retail sales, new export order component of the PMI survey, initial unemployment insurance claims, quarterly GDP growth
    \end{enumerate}
    \medskip
    \item {Estimation} sample: 1973m1-2020m5
\end{itemize}

\begin{figure}
     \only<1>{\includegraphics[width=0.65\linewidth,keepaspectratio=true]{FiguresPhillyFed/slides_data_FF.pdf}}\par
     \only<2>{\includegraphics[width=0.65\linewidth,keepaspectratio=true]{FiguresPhillyFed/slides_data_MF.pdf}}
\end{figure}

\end{frame}

%-----------------------------------------------------------------

\begin{frame}{Monthly GDP estimate}
\begin{itemize}
\item DFM also provides real-estimate of monthly GDP $\rightarrow$ assesment of buildup of risks at high frequency
\bigskip

\item Monthly GDP tracks well other existing measures:
\begin{itemize}
  \medskip
  \item \cite{stock1989new}, IHS-Markit, \cite{lewis2020us}
\end{itemize}

\end{itemize}

\begin{figure}
     \includegraphics[width=0.65\linewidth,keepaspectratio=true]{FiguresPhillyFed/slides_data_GDP.pdf}
\end{figure}

\end{frame}

%-----------------------------------------------------------------

\begin{frame}{Markov-switching model captures GAR...}
\vspace*{-0.25in}
\begin{eqnarray*}
\bar{\Delta} {y}_{t+1,t+12} &=& \rr{\alpha_y (s_t)} + \pb{\beta_y (s_t)} f_t + \pb{\gamma_y (s_t)} m_t + \rr{\sigma_y (s_t)} \varepsilon^y_t
\end{eqnarray*}
\begin{itemize}
\item \rr{Negative correlation between mean and volatility}
\bigskip
\small
\begin{table}[ht!]
  \centering
  \vspace{-0.25cm}
  \begin{tabular}{p{1.5cm} p{1cm} p{2cm} p{1cm} p{2cm}}
\hline
& \multicolumn{2}{c}{\textbf{Bad Regime}} & \multicolumn{2}{c}{\textbf{Normal Regime}} \\
    \hline
  $\alpha_y(s_t)$   & -0.99 & [-1.25,-0.75] & \h0.59 & [\h0.48,\h0.68]  \\
  $\sigma_y(s_t)$   & \h2.60 & [\h2.35,\h2.89] & \h0.64 & [\h0.57,\h0.71]  \\
  \midrule
    \midrule
    \end{tabular}%
\end{table}%

\item \pb{GDP growth more sensitive to fundamentals in bad regime}
\bigskip

\begin{table}[ht!]
  \centering
  \vspace{-0.25cm}
  \begin{tabular}{p{1.5cm} p{1cm} p{2cm} p{1cm} p{2cm}}
\hline
& \multicolumn{2}{c}{\textbf{Bad Regime}} & \multicolumn{2}{c}{\textbf{Normal Regime}} \\
    \hline
  $\beta_y(s_t)$    & -0.29 & [-0.49,-0.08] & -0.02 & [-0.08,\h0.04]  \\
  $\gamma_y(s_t)$   & \h0.54 & [\h0.22,\h0.91] & \h0.29 & [\h0.15,\h0.41]  \\
  \midrule
    \midrule
    \end{tabular}%
\end{table}%
\end{itemize}

\normalsize
\begin{itemize}
     \item Asymmetry: GDP growth more sensitive to macro than to financial conditions in  \textit{bad regime}
\end{itemize}

\end{frame}

%-----------------------------------------------------------------

\begin{frame}{GAR also operates through transition probabilities}
\begin{itemize}
\item Similar response of transition probabilities to FF and MF
\end{itemize}
\begin{figure}
    \includegraphics[width=0.45\linewidth,keepaspectratio=true]{FiguresPhillyFed/SVAR_StateDependent_2Regimes_Pr12.pdf}
    \includegraphics[width=0.45\linewidth,keepaspectratio=true]{FiguresPhillyFed/SVAR_StateDependent_2Regimes_Pr21.pdf}
\end{figure}

\begin{itemize}
\item Transition probability to \textit{bad regime}:
\end{itemize}
\begin{figure}
    \includegraphics[width=0.85\linewidth,keepaspectratio=true]{FiguresPhillyFed/SVAR_StateDependent_2Regimes_Pr12_fitted.pdf}
\end{figure}


\end{frame}

%-----------------------------------------------------------------

\begin{frame}{The predictive distribution of GDP growth}
\vspace*{-0.5in}
%\begin{eqnarray*}
%\bar{\Delta} {y}_{t+1,t+12} &=& \alpha_y (s_t) + \beta_y (s_t) f_t + \gamma_y (s_t) m_t + \sigma_y (s_t) \varepsilon^y_t
%\end{eqnarray*}

\begin{align*}
p\left(\bar{\Delta} y_{t+1,t+H}|\mathcal{I}_t\right) &= \int_\theta \int_{\epsilon^y_t} \left[ \int_{s_{t-H+1:t}} p(\bar{\Delta} y_{t+1,t+H}, s_{t-H+1:t} | \mathcal{I}_t, s_{1:t-H+1},\theta) ds_{t-H+1:t} \right] \\ &  \times p(\epsilon^y_t|\mathcal{I}_t,\theta) p(\theta|\mathcal{I}_t) d \epsilon^y_t  d \theta
\end{align*}

\begin{itemize}
\item Sources of uncertantinty:
  \begin{enumerate}
    \medskip
    \item Parameter uncertainty $p(\theta|\mathcal{I}_t)$
    \medskip
    \item Shock uncertaninty $p(\epsilon^y_t|\mathcal{I}_t,\theta)$
    \medskip
    \item Regime uncertainty $p(\bar{\Delta} y_{t+1,t+H}, s_{t-H+1:t} | \mathcal{I}_t, s_{1:t-H+1},\theta)$
  \end{enumerate}
\bigskip
\item Challenge for real-time inference: $\mathcal{I}_t$ does not contain $y_{t+1},\dots \y_{t+H} \rightarrow \bar{\Delta} y_{t-11,t-12+H}, \dots $ unobserved
\end{itemize}

\end{frame}


%-----------------------------------------------------------------

\begin{frame}{Estimated regimes}
\medskip
\begin{itemize}
\item Uncertainty about $s_t$ through direct simulation of the Markov-chain
\end{itemize}

\vspace*{-0.25cm}
\begin{figure}
     \includegraphics[width=0.85\linewidth,keepaspectratio=true]{FiguresPhillyFed/SVAR_StateDependent_2Regimes_PrReg2.pdf}
\end{figure}

\end{frame}


%-----------------------------------------------------------------

\begin{frame}{The evolution of Growth-at-Risk}
\begin{itemize}
\item MS model captures asymmetric dynamics of conditional quantiles
\item Measures of vulnerability: \textbf{SF} and \textbf{LR} positively correlated and \textbf{SF} more volatile
\end{itemize}
\centering
\begin{figure}
    \includegraphics[width=0.85\linewidth,keepaspectratio=true]{FiguresPhillyFed/SVAR_StateDependent_2Regimes_Quantiles_MS.pdf}
\end{figure}

\end{frame}

%-----------------------------------------------------------------

\begin{frame}{Quantile regression model}

\begin{eqnarray*}
\widehat{\mathcal{Q}}_{\tau}(\bar{\Delta} {y}_{t+1,t+12}|x_t) &=& \hat{\alpha}_{\tau} + \hat{\beta}_{\tau} f_t + \hat{\gamma}_{\tau} m_t, \label{eq:GAR}
\end{eqnarray*}

\begin{itemize}
\item $\Delta {y}_{t+1,t+12}$ is calculated from our monthly GDP series
\medskip
\item Smooth quantile function \citep{AzzaliniCapitanio2003}
\medskip
\item $\hat{\alpha}_\tau$, $\hat{\beta}_\tau$ and $\hat{\gamma}_\tau$ fold all the mechanims of GAR
  \begin{itemize}
    \medskip
     \item Lower quantile with similar growth than MS \textit{bad regime}
       \medskip
     \item Lower quantile more responsive to $f_t$ and $m_t$
       \medskip
       \item Assymetry in $m_t$
  \end{itemize}
\end{itemize}

\vspace*{-0.25cm}
\small
\begin{table}[ht!]
  \centering
  \begin{threeparttable}
  %\caption*{\textbf{Estimated Coefficients of Markov-Switching and Quantile Regression Models}}\label{tab:MS-QR}
    \begin{tabular}{p{0.5cm} p{1.cm} p{1.8cm} p{1.cm} p{1.8cm} p{1.cm} p{1.8cm} }
      \midrule
      & \multicolumn{6}{c}{\textbf{Quantile Regression}} \\
          \cline{2-7}
     & \multicolumn{2}{c}{\textbf{\textit{25th Quantile}}} & \multicolumn{2}{c}{\textbf{\textit{Median}}} & \multicolumn{2}{c}{\textbf{\textit{75th Quantile}}} \\
     \midrule
     $\alpha_{\tau}$ & -0.99  & [-1.11,-0.86] & \h0.20  & [\h0.11,\h0.29] &  \h1.02 & [\h0.96,\h1.08]\\
$\beta_{\tau}$  & -0.60  & [-0.71,-0.48]  & -0.29  & [-0.37,-0.21] & -0.09 & [-0.14,-0.04] \\
$\gamma_{\tau}$  & \h0.68  & [\h0.42,\h0.94] & \h0.38  & [\h0.21,\h0.55] &  \h0.33 & [\h0.20,\h0.46] \\
    \midrule
    \midrule
    \end{tabular}%
    \end{threeparttable}
  \label{tab:MS_QR_coeff}%
\end{table}%
\normalsize

\end{frame}

%-----------------------------------------------------------------

\begin{frame}{Markov-switching vs Quantile regressions}
\begin{itemize}
\item MS and QR models capture similar dynamics.
\medskip
\begin{itemize}
  \item Lower quantile exibits more volatile than upper quantile \citep{ABG19}
    \medskip
  \item In MS model, lower quantile associated with transition to \textit{bad regime}
\end{itemize}

\end{itemize}
\begin{figure}
    \includegraphics[width=0.85\linewidth,keepaspectratio=true]{FiguresPhillyFed/SVAR_StateDependent_2Regimes_Quantiles_MS_QR.pdf}
\end{figure}
\end{frame}

%-----------------------------------------------------------------

\begin{frame}{Risk assessment during the GFC}

\begin{figure}
    \includegraphics[width=0.95\linewidth,keepaspectratio=true]{FiguresPhillyFed/SVAR_StateDependent_2Regimes_RiskAssessment_GFC.pdf}
\end{figure}
\end{frame}

%-----------------------------------------------------------------

\begin{frame}{COVID-19 build-up of risk}

\begin{itemize}
\item Condition on filtered regime as of March-2019
\medskip
\item \textbf{March-13}: Financial turmoil $\rightarrow$ increase downside risk
\medskip
\item \pb{\textbf{April-2}}: Init u-claims rose $\sim7$ million $\rightarrow$ $\textit{bad regime}$
\end{itemize}
\begin{figure}
    \includegraphics[width=0.65\linewidth,keepaspectratio=true]{FiguresPhillyFed/SVAR_StateDependent_2Regimes_Density_Mar2020.pdf}
\end{figure}
\end{frame}



%-----------------------------------------------------------------

\begin{frame}{Conclusions}
\end{frame}


%-----------------------------------------------------------------

\frame{
\frametitle{References}
\bibliographystyle{ecca}
\bibliography{GarRef}
}


%-----------------------------------------------------------------
% BLANK SLIDE
\begin{frame}

\end{frame}

%-----------------------------------------------------------------
\begin{frame}{Motivation}
\vspace*{0.12in}
 \begin{itemize}
     \item GDP growth features time-variation in mean and volatility
 \medskip
     \begin{itemize}
         \item Captured using regime-change models
     \end{itemize}
    \bigskip
     \item \cite{ABG19}: connection between conditional distribution of future GDP growth and fundamentals
     \bigskip
     \item \rr{Growth-at-Risk (GAR)}: Conditional mean and volatility are negatively correlated
\end{itemize}

\begin{figure}
     \includegraphics[width=0.65\linewidth,keepaspectratio=true]{Figures/univariatefull-VC.pdf}
\end{figure}
\end{frame}

%-----------------------------------------------------------------

\begin{frame}{Example of downside and upside risk in MS model}

\begin{figure}
    \includegraphics[width=0.85\linewidth,keepaspectratio=true]{FiguresPhillyFed/SVAR_StateDependent_2Regimes_KernelDensity_2008_2018.pdf}
\end{figure}

\end{frame}



\end{document}
